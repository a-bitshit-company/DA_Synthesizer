\chapter*{Kurzfassung /Abstract }
\label{cha:abstract}

Die vorliegende Diplomarbeit beschäftigt sich mit subtraktiver Klangsynthese mittels elektronischer Hardware am Beispiel eines selbstgebauten modularen Synthesizers. \\

Der dokumentierte Synthesizer stellt ein in sich geschlossenes System dar, welches jedoch bei Bedarf mit externen Komponenten erweitert werden kann. Zu diesem Zweck streben wir Kompatibilität mit dem Eurorack Format an, welches einen de-facto Standard für modulare elektronische Synthesizer darstellt. \\ \\


This thesis deals with analog sound synthesis through electronic hardware using the example of a home-made modular synthesizer system. \\

The documented Synthesizer is a self-contained system, which can be extended by external components. To this end, we strive for compatibility with the eurorack format, which represents a de-facto standard for modular electronic synthesizer systems. \\ \\

Gliederung des Abstract in \textbf{Thema}, \textbf{Ausgangspunk}, \textbf{Kurzbeschreibung}, \textbf{Zielsetzung}.  

\subparagraph{Projektergebnis}

Allgemeine Beschreibung, was vom Projektziel umgesetzt wurde, in einigen kurzen Sätzen. Optional Hinweise auf Erweiterungen. Gut machen sich in diesem Kapitel auch Bilder vom Gerät (HW) bzw. Screenshots (SW).
Liste aller im Pflichtenheft aufgeführten Anforderungen, die nur teilweise oder gar nicht umgesetzt wurden (mit Begründungen).
