%%%% Time-stamp: <2013-02-25 10:31:01 vk>


\chapter*{Kurzfassung /Abstract }
\label{cha:abstract}

Eine Kurzfassung ist in deutscher sowie ein Abstract in englischer Sprache mit je maximal einer A4-Seite zu erstellen. Die Beschreibung sollte wesentliche Aspekte des Projektes in technischer Hinsicht beschreiben. Die Zielgruppe der Kurzbeschreibung sind auch Nicht-Techniker! Viele Leser lesen oft nur diese Seite. \\ \\

Beispiel für ein Abstract (DE und EN) \\ \\
Die vorliegende Diplomarbeit beschäftigt sich mit verschiedenen Fragen des Lernens Erwachsener – mit dem Ziel, Lernkulturen zu beschreiben, die die Umsetzung des Konzeptes des Lebensbegleitenden Lernens (LBL) unterstützen. Die Lernfähigkeit Erwachsener und die unterschiedlichen Motive, die Erwachsene zum Lernen veranlassen, bilden den Ausgangspunkt dieser Arbeit. Die anschließende Auseinandersetzung mit Selbstgesteuertem Lernen, sowie den daraus resultierenden neuen Rollenzuschreibungen und Aufgaben, die sich bei dieser Form des Lernens für Lernende, Lehrende und Institutionen der Erwachsenenbildung ergeben, soll eine erste Möglichkeit aufzeigen, die zur Umsetzung dieses Konzeptes des LBL beiträgt. Darüber hinaus wird im Zusammenhang mit selbstgesteuerten Lernprozessen Erwachsener die Rolle der Informations- und Kommunikationstechnologien im Rahmen des LBL näher erläutert, denn die Eröffnung neuer Wege zur orts- und zeitunabhängiger Kommunikation und Kooperation der Lernenden untereinander sowie zwischen Lernenden und Lernberatern gewinnt immer mehr an Bedeutung. Abschließend wird das Thema der Sichtbarmachung, Bewertung und Anerkennung des informellen und nicht-formalen Lernens aufgegriffen und deren Beitrag zum LBL erörtert. Diese Arbeit soll einerseits einen Beitrag zur besseren Verbreitung der verschiedenen Lernkulturen leisten und andererseits einen Reflexionsprozess bei Erwachsenen, die sich lebensbegleitend weiterbilden, in Gang setzen und sie somit dabei unterstützen, eine für sie geeignete Lernkultur zu finden. \\ \\


This thesis deals with the various questions concerning learning for adults – with the aim to describe learning cultures which support the concept of live-long learning (LLL). The learning ability of adults and the various motives which lead to adults learning are the starting point of this thesis. The following analysis on self-directed learning as well as the resulting new attribution of roles and tasks which arise for learners, trainers and institutions in adult education, shall demonstrate first possibilities to contribute to the implementation of the concept of LLL. In addition, the role of information and communication technologies in the framework of LLL will be closer described in context of self-directed learning processes of adults as the opening of new forms of communication and co-operation independent of location and time between learners as well as between learners and tutors gains more importance. Finally the topic of visualisation, validation and recognition of informal and non-formal learning and their contribution to LLL is discussed. \\

Gliederung des Abstract in \textbf{Thema}, \textbf{Ausgangspunk}, \textbf{Kurzbeschreibung}, \textbf{Zielsetzung}.  

\subparagraph{Projektergebnis}

Allgemeine Beschreibung, was vom Projektziel umgesetzt wurde, in einigen kurzen Sätzen. Optional Hinweise auf Erweiterungen. Gut machen sich in diesem Kapitel auch Bilder vom Gerät (HW) bzw. Screenshots (SW).
Liste aller im Pflichtenheft aufgeführten Anforderungen, die nur teilweise oder gar nicht umgesetzt wurden (mit Begründungen).