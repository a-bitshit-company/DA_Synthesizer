\chapter{Einleitung}

Die meisten Aufgaben bei der Produktion moderner Musik werden heutzutage am Computer erledigt. Doch während digitale Werkzeuge viele dieser Aufgaben zufriedenstellend erledigen können, liegt für viele Musiker ein besonderer Reiz in der Soundproduktion mit analoger Hardware. Diese Einstellung könnte daran liegen, dass analoge Instrumente einen eigenen Charakter mit sich bringen und für viele auch intuitiver zu handhaben sind. Besonders modulare Synthesizer bieten hier eine Plattform, welche die Welt der Klangsynthese greifbar und zugänglich macht.

Die größte Stärke modularer Synthesizer ist ihre unermessliche Flexibilität. Während bei herkömmlichen Synthesizern viele oder sogar alle Signalwege fest verkabelt sind und im Inneren des Instrumentes versteckt werden, kann bei modularen Synthesizern frei entschieden werden, welche Teile des Synthesizers sich auf welche Art und Weise gegenseitig beeinflussen. So kann man den Aufbau und die Struktur seines Instrumentes beliebig und flexibel verändern, sogar während man es benutzt, beispielsweise während eines Auftritts. Module aus verschiedensten Quellen können miteinander verwendet werden, solange sie den selben Standards folgen. Modulare Synthesizer sind ein ideales Werkzeug für Klangtechnische Experimente, welche nur von der vorhandenen Hardware und von der eigenen Fantasie beschränkt sind.

Da Module von etablierten Herstellern meist mit einem beträchtlichen finanziellen Aufwand verbunden sind, wollen wir die benötigten Module mit herkömmlichen Bauteilen selbst fertigen. Dabei stützen wir uns auf Öffentliche Anleitungen, Schaltkreise und auf die Dokumentation des A-100 von Doepfer \cite{doepfer:A-100}.

\pagebreak

Der resultierende Synthesizer soll in der Lage sein, eine simple klassische Signalverarbeitungskette für subtraktive Klangsynthese (beschrieben in Abschnitt \ref{subKS}) bereitzustellen. Zu diesem Zweck werden Oszillatoren oder ähnliche Klangquellen, ein \ac{VCA}, ein Hüllkurvengenerator, und eine Kontrollspannungsquelle für diesen Hüllkurvengenerator (Im Idealfall ein Sequencer) benötigt. Auch sollen alle Bestandteile dieses Synthesizers den Standards des Eurorack-Formates entsprechen, um mit Eurorack-Modulen kompatibel sein.

\section{Vertiefende Aufgabenstellung}

\subsection{Felix Perktold}
\begin{itemize}
\item 20\% Recherche
\item 60\% Dokumentation der Theoretischen Grundlagen, geschichtliche Hintergründe
\item 20\% Konstruktion der Module
\end{itemize}

\subsection{Matteo Kastler}
\begin{itemize}
\item 20\% Recherche praktische elektronische Umsetzung
\item 20\% Dokumentation 
\item 60\% Konstruktion der Module
\end{itemize}
