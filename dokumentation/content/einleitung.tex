\chapter{Einleitung}

Die meisten Aufgaben bei der Produktion moderner Musik werden heutzutage am Computer erledigt. Doch während digitale Werkzeuge viele dieser Aufgaben zufriedenstellend erledigen können, besitzt Soundproduktion mit analoger Hardware für viele Musiker einen besonderen Reiz. Diese Einstellung könnte daran liegen, dass analoge Instrumente eine eigenen Charakter mit sich bringen und für viele auch intuitiver zu handhaben sind. Besonders modulare Synthesizer bieten hier eine Plattform, welche die Welt der Klangsynthese greifbar und zugänglich macht.

Die größte Stärke modularer Synthesizer ist ihre unermessliche Flexibilität. Während bei herkömmlichen Synthesizern viele oder sogar alle Signalwege fest verkabelt sind und im inneren des Instrumentes versteckt werden, kann man bei modularen Synthesizern frei entscheiden, welche Module sich auf welche Art und Weise gegenseitig beeinflussen. So kann man den Aufbau und die Struktur seines Instrumentes beliebig und flexibel verändern, sogar während man es benutzt, also beispielsweise während einem Auftritt. Module aus verschiedensten Quellen können miteinander verwendet werden, solange sie den selben Standards folgen. Modulare Synthesizer sind ein ideales Werkzeug für Klangtechnische Experimente, welche nur von der vorhandenen Hardware und von der eigenen Fantasie beschränkt sind.

Da Module von etablierten Herstellern meist mit einem beträchtlichen finanziellen Aufwand verbunden sind, wollen wir die benötigten Module mit herkömmlichen Bauteilen selbst fertigen. Dabei stützen wir uns auf Öffentliche Anleitungen, Schaltkreise und auf die Dokumentation des A-100 von Doepfer \cite{doepfer:A-100}.

Der resultierende Synthesizer soll in der Lage sein, eine simple klassische Signalverarbeitungskette für subtraktive Klangsynthese (beschrieben in Abschnitt \ref{subKS}) bereitzustellen. Zu diesem Zweck werden Oszillatoren oder ähnliche Klangquellen, ein \ac{VCA}, ein Hüllkurvengenerator, und eine Kontrollspannungsquelle für diesen Hüllkurvengenerator (Im Idealfall ein Sequencer) benötigt. Auch sollen alle Bestandteile dieses Synthesizers den Standards des Eurorack-Formates entsprechen, um mit Eurorack-Modulen kompatibel sein.

\section{Vertiefende Aufgabenstellung}

\subsection{Felix Perktold}
Recherche, Dokumentation der Theoretischen Grundlagen, geschichtlichen Hintergründe

\subsection{Matteo Kastler}
20\% Recherche praktische elektronische Umsetzung
20\% Dokumentation 
60\% Konstruktion der Module

\section{Dokumentation der Arbeit}

Es werden folgende Projektergebnisse dokumentiert:

\begin{itemize}
	\item Theoretische Grundlagen
	\item Praktische Umsetzung
	\item Ergebnisse
\end{itemize}

Weitere Anregungen:

\begin{itemize}
	\item Fertigungsunterlagen
	\item Testfälle (Messergebnisse…)
	\item Benutzerdokumentation
	\item Verwendete Technologien und Entwicklungswerkzeuge
\end{itemize}
