\chapter{Einleitung}

Da vorgefertigte Module meist mit einem beträchtlichen finanziellen Aufwand verbunden sind, versuchen wir die Module mit herkömmlichen Bauteilen selbst zu fertigen.

Der resultierende Synthesizer soll in der Lage sein, eine simple klassische Signalverarbeitungskette für subtraktive Klangsynthese bereitzustellen. Auch soll dieser Synthesier im Idealfall den Standards des Eurorack-Formates entsprechen, jedoch mindestens mit Eurorack Modulen kompatibel sein.

\section{Vertiefende Aufgabenstellung}

\subsection{Schüler*innen Name 1}

\subsection{Schüler*innen Name 2}

\section{Dokumentation der Arbeit}

Es werden die Projektergebnisse dokumentiert

\begin{itemize}
	\item Grundkonzept
	\item Theoretische Grundlagen
	\item Praktische Umsetzung
	\item Lösungsweg
	\item Alternativer Lösungsweg
	\item Ergebnisse inkl. Interpretation
\end{itemize}

Weitere Anregungen:

\begin{itemize}
	\item Fertigungsunterlagen
	\item Testfälle (Messergebnisse…)
	\item Benutzerdokumentation
	\item Verwendete Technologien und Entwicklungswerkzeuge
\end{itemize}
