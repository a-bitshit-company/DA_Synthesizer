\chapter{Einleitung}

Eine weitere Eigenschaft modularer Synthesizer ist ihre unermessliche Flexibilität. Während bei herkömmlichen Synthesizern viele oder sogar alle Signalwege fest verkabelt und verlötet sind und im inneren des Instrumentes versteckt werden, kann man bei modularen Synthesizern frei entscheiden, welche Module sich auf welche Art und Weise gegenseitig beeinflussen. So kann man den Aufbau und die Struktur seines Instrumentes beliebig und flexibel verändern, sogar während man es benutzt, also beispielsweise während einem Auftritt. Auch können Module aus verschiedensten Quellen miteinander benutzt werden, solange sie den selben Standards, wie beispielsweise denen des Eurorack Formates, folgen. Modulare Synthesizer sind ein ideales Werkzeug für Klangtechnische Experimente, welche nur von der vorhandenen Hardware und von der eigenen Fantasie beschränkt sind.

Da Module von etablierten Herstellern meist mit einem beträchtlichen finanziellen Aufwand verbunden sind, wollen wir die benötigten Module mit herkömmlichen Bauteilen selbst fertigen. Dabei stützen wir uns auf Öffentliche Anleitungen und Schaltkreise und auf die Dokumentation des A-100 von Doepfer und nutzen die Anlagen und Einrichtungen der HTL-Anichstraße, wie beispielsweise Arbeitspläze mit Lötstationen, Oszilloskopen und einige Elektronische Bauteile aus dem Magazin. Da die HTL-Anichstraße das Perfekte Umfeld bietet, ein solches Projekt durchzuführen, haben wir uns dazu entschieden dies als Diplomarbeitsprojekt zu tun.

Der resultierende Synthesizer soll in der Lage sein, eine simple klassische Signalverarbeitungskette für subtraktive Klangsynthese bereitzustellen. Zu diesem Zweck werden Oszillatoren oder ähnliche Klangquellen, ein \ac{VCA}, ein Hüllkurvengenerator, und eine Kontrollspannungsquelle für diesen Hüllkurvengenerator (Im Idealfall ein Sequencer) benötigt. Auch sollen alle Bestandteile dieses Synthesizers den Standards des Eurorack-Formates entsprechen, um mit Eurorack-Modulen kompatibel sein.

\section{Vertiefende Aufgabenstellung}

\subsection{Felix Perktold}

\subsection{Matteo Kastler}

\section{Dokumentation der Arbeit}

Es werden folgende Projektergebnisse dokumentiert:

\begin{itemize}
	\item Theoretische Grundlagen
	\item Praktische Umsetzung
	\item Ergebnisse
\end{itemize}

Weitere Anregungen:

\begin{itemize}
	\item Fertigungsunterlagen
	\item Testfälle (Messergebnisse…)
	\item Benutzerdokumentation
	\item Verwendete Technologien und Entwicklungswerkzeuge
\end{itemize}
