\section{Das Eurorack Format}
\label{sec:orgc35feee}

1996 veröffentlichte Doepfer Musikelektronik GmbH eine Reihe an Synthesizermodulen. Schnell wurden kompatible Module von anderen Herstellern hergestellt, wodurch das Eurorack Format zum de-facto Standard für modulare Synthesizer wurde. Heute gibt es tausende von Eurorack Modulen, hergestellt von bekannten Firmen wie Moog, Roland, Behringer, auf Eurorack spezialisierten Herstellern wie Make Noise und es gibt eine lebendige DIY-szene mit vielen öffentlichen Designs, Anleitungen, Schematics, vorbereiteten Kits zum zusammenbauen und mehr.

Essentiell bei Eurorack Modulen ist, dass viele Funktionen nicht nur durch den Benutzer (durch Knöpfe, Potentiometer, etc) sondern auch durch andere Module mithilfe von sog. Kontrollspannung (CV) ansteuerbar sind. So kann z.B die Frequenz eines Oszillators, der Cutoff eines Filters, Attack und Releaselänge eines Envelopes usw. durch ein anderes Signal kontrolliert werden; Diese Kontrollspannung kann wiederum aus verschiedensten Modulen wie z.B. einem MIDI Interface, einem LFO oder sogar einem anderen Audiosignal kommen. Die Module sind nicht fest verkabelt, sondern werden vom Benutzer "on the fly" mit Patchkabeln (3.5mm mono) verbunden. Dadurch entsteht ein Netzwerk an elektronischen Schaltungen welche sich gegenseitig beeinflussen und hochschaukeln, was zu idealerweise wohlklingenden, jedoch in jedem Fall interessanten Effekten führt.

\subsection{Vokabular}
\label{sec:orgbb4ceab}
\subsubsection{Kontrollspannung, Control Voltage}
\label{sec:orgfd3fe14}
Kontrollspannung (CV) ist das Herz des Konzepts eines Modularen Synthesizer. Während normale Synthesizer wie der Moog mini 
