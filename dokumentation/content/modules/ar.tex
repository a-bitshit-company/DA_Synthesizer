% Created 2023-01-25 Wed 15:14
% Intended LaTeX compiler: pdflatex
\documentclass[11pt]{article}
\usepackage[utf8]{inputenc}
\usepackage[T1]{fontenc}
\usepackage{graphicx}
\usepackage{longtable}
\usepackage{wrapfig}
\usepackage{rotating}
\usepackage[normalem]{ulem}
\usepackage{amsmath}
\usepackage{amssymb}
\usepackage{capt-of}
\usepackage{hyperref}
\date{\today}
\title{}
\hypersetup{
 pdfauthor={},
 pdftitle={},
 pdfkeywords={},
 pdfsubject={},
 pdfcreator={Emacs 28.2 (Org mode 9.5.5)}, 
 pdflang={English}}
\begin{document}

\tableofcontents

\section{Attack-Release Hüllkurvengenerator}
\label{sec:org0001e10}
Der AR Hüllkurvengenerator stellt eine \href{file:///home/felixp/Documents/diplomarbeit/dokumentation/content/theoretische\_grundlagen.org}{Hüllkurve} bereit, wenn an seinem Kontrollspannungseingang eine Spannung von etwa \SI{5}{\volt} anliegt. AR steht dabei für Attack-Release, das AR-Modul stellt somit eine einfache Variante des klassischen ADSR-Hüllkurvengenerators dar.
\subsection{Spezifikationen}
\label{sec:org10bedb0}
\subsection{Elektronik}
\label{sec:orga44b8a3}
\subsection{Schematics}
\label{sec:org639e78e}
\subsection{Benutzung}
\label{sec:orgd03db95}
\end{document}