\section{Mixer}
\label{sec:org6ef1de2}
\subsection{Einleitende Beschreibung}
\label{sec:orgefd4efb}
Die Aufgabe eines Mixers ist die Zusammenführug (das "mixen") von mehreren (in diesem Fall bis zu drei) Signalen. Unser Mixer Modul besitzt neben den regulären auch invertierende inputs, welche das Inputsignal vom Outputsignal abziehen anstatt es zu addieren.
\subsection{Spezifikationen}
\label{sec:orgf7e4855}
Spannung: voller Spannungsbereich möglich (=> bis zu 24\href{file:///home/felixp/Documents/diplomarbeit/dokumentation/content/hauptteil.org}{Vpp})
\subsection{Elektronik}
\label{sec:org63838fc}
\subsection{Schematics}
\label{sec:org9cf894e}
\subsection{Benutzung}
\label{sec:orga527c50}
Die zu mixenden Signale werden durch die oberen drei Audiobuchsen angeschlossen. Die unterste Audiobuchse liefert das Ausgangssignal. Als einfachen Patch könnte man die beiden Signale des \href{oscillator.org}{Oszillator x2} Moduls zusammenführen um beide Oszillatoren auf einmal zu hören.
