\chapter{Praktische Umsetzung}

\section{Gehäuse}
\label{sec:orge13c608}
Das Gehäuse muss so dimensioniert sein, dass es Eurorack-kompatible Module beherrbergen kann. Wichtig ist dabei vor allem der vertikale Abstand der Schienen, auf welchen die Module befestigt werden. Dieser misst laut dem \emph{Doepfer System A100 Handbuch Seite 3} 3 HE (Höheneinheiten). 1 HE beträgt dabei \SI{44.45}{\milli\meter}. Die Breite der Module beträgt ein Vielfaches von \SI{5.08}{\milli\meter}, beziehungsweise 0.2 Zoll. Manche Gehäuse, wie beispielsweise die von der Firma Intellijel hergestellten, können auch Module welche nur eine Höheneinheit hoch sind beherrbergen.

\section{Spannungsquelle}
\label{sec:org994e063}
Eurorack Synthesizer arbeiten mit Spannungen von \SI{24}{\volt} peak to peak, benötigen also eine Spannungsquelle, welche \SI{-12}{\volt} bis \SI{+12}{\volt} bereitstellen kann. Manche Module benötigen außerdem eine Seperate \SI{5}{\volt} Leitung um zum Beispiel Microcontroller zu betreiben. Eine beliebte Wahl in der DIY-Szene ist die Mean Well RT65B PSU, da sie \SI{\pm 12}{\volt} und \SI{5}{\volt} mit einer genügenden maximalen Stromstärke/Leistung bereitstellt um ein Anfängersystem zu versorgen und verhältnismäßig günstig ist.

\section{Spannungsverteilung}
\label{sec:orgcc96145}
Die Module werden über ein 10 Pin IDC-Flachbandkabel mit Strom versorgt. Dabei sind die mittleren 6 Pins durchverbunden und geerdet. Die äußeren vier Pins sind paarweise verbunden und liefern jeweils \SI{+12}{\volt} und \SI{-12}{\volt}. Der PIN, welcher auf \SI{-12}{\volt} liegt, ist üblicherweise rot gekennzeichnet.

\section{Transistoren abgleichen}
\label{sec:org617748e}
\url{https://www.alldatasheet.com/view.jsp?Searchword=BC548}
Manchmal werden Transistoren benötigt, welche einen möglichst gleichen internen Verstärkungsfaktor besitzen. Bereits aufeinander abgestimmte Transistorpaare sind zwar im Handel erhältlich, jedoch ist es einfach, das Abgleichen von Transistoren selbst vorzunehmen. Baut man den zu messenden Transistor in den folgenden Schaltkreis ein und legt man bei \(V_{CC}\) eine Spannung an, kann man seinen internen Verstärkungsfaktor herausfinden, indem man das Potential zwischen Punkt A (beim Kollektor des Transistors) und der Masse misst:

\begin{circuitikz}[european]
\draw{
(0,0) -- ++(3,0) node[ground]{}
(0,0) to[R, l=\SI{22}{\kilo\ohm}] ++(0,2.5) coordinate (T) to[R, l=\SI{220}{\kilo\ohm}] ++(0,2.5) -- ++(3,0) node[vcc]{$V_{CC}$} coordinate (U)
(T) -- ++(2,0) node[npn, tr circle](Q){}
(Q.C) to[R, l_=\SI{10}{\kilo\ohm}] (Q.C|-U)
(Q.E) to[R, l=\SI{1}{\kilo\ohm}] (Q.E|-0,0)
(Q.C) node[right] {A}
};
\end{circuitikz}

Diese Transistorenpaare sollten auf der Platine in unmittelbarer Nähe zueinander sein, da die Temperatur einen Einfluss auf die Eigenschaften eines Transistors hat.

\section{Module}
\label{sec:org868cbd5}

Im Folgenden werden die Module, welche den Synthesizer ausmachen, beschrieben. Alle dieser Module bestehen aus einem Panel, welches als User Interface dient und einer Platine, welche mit den elektronischen Komponenten bestückt ist. Panels besitzen mindestens eine Audiobuchse um CV, Audio, Triggersignale und andere Arten von Spannungssignalen auszugeben und oft eine beliebige Auswahl an Audioeingängen für Kontrollspannungen, Audio-Inputs und ähnlichem und andere Interfacekomponenten wie Potentiometern, Schaltern, Knöpfen, LEDs und weiterem.

Die elektronischen Komponenten können durch verschiedene Methoden zusammengeschalten werden, Beispiele dafür sind:
\begin{itemize}
\item Breadboards:
vor allem geeignet zum erstellen von Prototypen
\item THT Platinen:
eine schnelle Methode um eine einzelne Platine zu fertigen
\item Selbst geätzte oder vorgefertigte Platinen:
eine Methode mit relativ geringem Fehlerpotential, ideal wenn eine größere Anzahl gleichartiger Platinen gefertigt werden sollen, beispielsweise für DIY-Kits
\item "Deadbug" Methode:
Eletronische Komponenten werden ohne Platine "point to point" miteinander verlötet. Resultiert meist in Spaghettiähnlichen Strukturen, kann bei gekonnter Ausführung jedoch in sehr ästhetischen Schaltkreisen resultieren.
\end{itemize}

Die elektronischen Komponenten unserer Module sind auf THT-Platinen gelötet. Diese Platine wird im rechten Winkel in der Mitte des Panels befestigt. Interface-Komponenten welche vom Benutzerpanel aus zugänglich sein sollen werden über längere Kabel und Schraubklemmen auf der Platine verbunden.

Als Material für Panels sind Bleche, Dünne Holz-/Plastikplatten oder ähnliches geeignet, zu bedenken ist dabei die 

\begin{itemize}
\item Dicke des Materials:
Zum Bestücken sollte eine bestimmte Dicke nicht überschritten werden (abhängig von den gewählten Potentiometern, Audiobuchsen, Schaltern und Knöpfen)
\item Bearbeitbarkeit:
Es müssen Löcher für Interfacekomponenten gebohrt oder gestanzt werden, und das Material muss zugeschnitten werden
\item Beschriftbarkeit:
Für einfachere Zugänglichkeit und für bessere Ästhetik sollten die Panels bemalt und/oder beschriftet werden
\end{itemize}

Wir benutzen eine dünne, schwarz lackierte Holzplatte als Material für unsere Panels, diese bekleben wir mit transparenter Folie welche mit weißem permanent Marker beschriftet werden kann.

\subsection{Oszillator x2}
\label{sec:orgc5ad91d}
\subsubsection{Einleitende Beschreibung}
\label{sec:orgb1cc4e2}
Das 2xSqr Modul ist ein simples Signalerzeugendes Modul, welches zwei voneinander unabhängige Rechteckswellen generiert. Es besitzt zwei Audiobuchsen am Panel an welchen die Spannung der generierten Wellen anliegt und vier Potentiometer als verstellbare Wiederstände, mit welchen jeweils Amplitude und Frequenz der beiden Oszillatoren angesteuert werden können. Diese Potentiometer können durch Vactrols ersetzt werden, um das Modul Spannungssteuerbar zu machen.

\begin{circuitikz}
\ctikzset{bipoles/oscope/waveform=square}
\draw{node[oscopeshape](){}};
\end{circuitikz}

\subsubsection{Spezifikationen}
\label{sec:org8d4bc06}
Oszillator 1:
\begin{itemize}
\item Spannung: bis zu 10\href{file:///home/felixp/Documents/diplomarbeit/dokumentation/content/hauptteil.org}{Vpp}
\item Frequenzbereich:
\end{itemize}

Oszillator 2:
\begin{itemize}
\item Spannung: bis zu 10\href{file:///home/felixp/Documents/diplomarbeit/dokumentation/content/hauptteil.org}{Vpp}
\item Frequenzbereich
\end{itemize}

\subsubsection{Elektronik}
\label{sec:org3f1a848}
Resistor-Capacitor type Oszillator

\subsubsection{Schematics}
\label{sec:orgc7d72f6}

Oscillator logisch:

\begin{circuitikz}[european]
\draw{
(0,0) node[ground, anchor=center, name=G]{}
to[cC, invert, name=C1] ++(0,1)
-- ++(0.5,0)
node[schmitt, anchor=in](S){} (S.out)
-- ++(0,1)
to[pR, l_=R1, name=R1] ++(0,1)
(R1.wiper) -- (R1.wiper -| C1)
-- (C1)

(R1.wiper -| C1)
-- ++(0,1)
-- ++(3,0)
-- ++(0,-2)
-- ++(1,0)

node[op amp, anchor=+](OA1){}
(OA1.out) -- ++(0,1.2)
coordinate (T) -- (T -| OA1.-) -- (OA1.-)

(OA1.out)
to[C, name=C2, l=C2] ++(1,0)
-- ++(0,-0.5)
to[R, name=R2, l=R2] ++(0,-1.5)
node[ground]{}
(C2) -- ++(1,0)

node[op amp, anchor=+](OA2){}
(OA2.out) -- ++(0,1.2)
coordinate (T) -- (T -| OA2.-) -- (OA2.-)

(OA2.out) ++(1,-2.5)
node[ground]{}
to[pR, name=R3, l_=R3] ++(0, 3.5)
-- ++(1,0)
++(0.55,0) node[draw]{OUT}
(R3.wiper) -- (R3.wiper -| OA2.out) -- (OA2.out)
};

\end{circuitikz}

2x sqr Oszillator Modul:

\begin{circuitikz}[european]
\draw{
%% DIPs
(0, -7) node[dipchip, num pins=14, rotate=0](OPAMP){TL074}
(0, 0) node[dipchip, num pins=14, rotate=0](ST){CD1406}


%% Input/Output Terminals
(-4, 5) node[circ, name=+12V]{\SI{12}{V}}
(-4.5, 4.25) node[circ, name=GND]{GND}
(-5, 3.5) node[circ, name=-12V]{\SI{-12}{V}}


%% Connections Schmitt trigger inputs to GND
(ST.pin 3) -- (ST.pin 3 -| ST.center) -- (ST.south)
(ST.pin 5) -- (ST.pin 5 -| ST.center) -- (ST.south)
(ST.pin 7) -- (ST.pin 7 -| ST.center) -- (ST.south)
(ST.pin 9) -- (ST.pin 9 -| ST.center) -- (ST.south)
(ST.pin 11) -- (ST.pin 11 -| ST.center) -- (ST.south)
(ST.south) -- ++(0, -1) coordinate (GND2) node[circ]{}


(ST.pin 1) -- ++(-0.6, 0)
to[pR, l_=R1A, name=R1A]
++(0, -1.5)
(ST.pin 2) -- (ST.pin 2 -| R1A.wiper) -- (R1A.wiper)
(ST.pin 1 -| R1A) node[circ]{} -- ++(-1, 0)
coordinate (T) -- (T |- OPAMP.pin 3) -- (OPAMP.pin 3)

(ST.pin 13) -- ++(0.6, 0)
to[pR, l=R1B, name=R1B, mirror]
++(0, -1.5)
(ST.pin 12) -- (ST.pin 12 -| R1B.wiper) -- (R1B.wiper)
(ST.pin 13 -| R1B) node[circ]{} -- ++(1,0)
coordinate (T) -- (T |- OPAMP.pin 12) -- (OPAMP.pin 12)


(OPAMP.pin 2) -- (OPAMP.pin 1) node[circ]{} to[C, name=C2A, l=C2A] ++(-1,0)
coordinate (T) node[circ]{} -- (T |- OPAMP.pin 5) -- (OPAMP.pin 5)
(T) to[R, name=R2A] (T |- GND2) node[circ]{} -- (GND2)
(OPAMP.pin 6) -- (OPAMP.pin 7)

(OPAMP.pin 13) -- (OPAMP.pin 14) node[circ]{} to[C, name=C2B, l_=C2B] ++(1,0)
coordinate (T) node[circ]{} -- (T |- OPAMP.pin 10) -- (OPAMP.pin 10)
(T) to[R, name=R2B] (T |- GND2) node[circ]{} -- (GND2)
(OPAMP.pin 9) -- (OPAMP.pin 8)


(+12V) -- ++(1, 0) coordinate (T) to[nos] (T -| ST.pin 14) -- (ST.pin 14)
(+12V) ++(0.8, 0) coordinate (T) -- (T |- OPAMP.pin 4) -- (OPAMP.pin 4)
(-12V) -- ++(8.2, 0) coordinate (T) -- (T |- OPAMP.pin 11) -- (OPAMP.pin 11)

(GND) -- (GND -| R1B) node[circ]{} to[cC, invert, name=C1A] (R1B |- ST.pin 13)
(GND -| R1A) node[circ]{} to[cC, invert, name=C1B] (R1A |- ST.pin 1)

(GND -| R1B) -- ++(2, 0) coordinate (T) -- (GND2 -| T) node[circ]{} -- (GND2)
(T) -- (T |- OPAMP.pin 9) to[pR, mirror, name=R3B] ++(0, -1) -- ++(0, -1) node[draw]{OUT}
(R3B.wiper) -- (R3B.wiper -| OPAMP.pin 9) node[circ]{}

(GND2 -| R2A) -- ++(-1.5, 0) coordinate (T)
-- (T |- OPAMP.pin 6) to[pR, name=R3A] ++(0, -1) -- ++(0, -1) node[draw]{OUT}
(R3A.wiper) -- (R3A.wiper -| OPAMP.pin 6) node[circ]{}
};
\end{circuitikz}

\subsubsection{Benutzung}
\label{sec:org9be838e}
Das Panel ist aufgeteilt in einen linken und rechten oszillator, alle Elemente auf einer Seite gehören zu jeweils einem Oszillator. Die oberen beiden Potentiometer dienen zur Steuerung der \href{file:///home/felixp/Documents/diplomarbeit/dokumentation/content/theoretische\_grundlagen.org}{Frequenz}, die unteren beiden dienen zur Steuerung der \href{file:///home/felixp/Documents/diplomarbeit/dokumentation/content/theoretische\_grundlagen.org}{Amplitude} des Signals. Die Audiobuchsen dienen als Output. Der Schalter links oben aktiviert das Modul, die Oszillatoren sind nicht seperat voneinander an/ausschaltbar.

\subsection{Low Frequency Oscillator}
\label{sec:orgf4b3a9c}
Ein Low Frequency Oscillator, kurz LFO, generiert ein Signal, welches in einem niedrigen Frequenzbereich oszilliert. Wir benutzen als Vorlage für unseren LFO den "Simple LFO" von David Haillant. Dieses Modul erzeugt langsam oszillierende Spannungen in Form einer Rechteckswelle und einer Dreieckswelle.
\subsubsection{Spezifikationen}
\label{sec:org22e61dc}
\subsubsection{Elektronik}
\label{sec:org09e4d6f}
Wir verzichten bei unserer Ausführung des Moduls auf die vorhergesehenen Leuchtdiode, welche einen visuellen Indikator für die Frequenz des Ausgangssignals bieten würde. Dadurch wird am verwendeten TL074 ein Operationsverstärker frei. Dieser wird als zweiter Puffer benutzt, um beide Wellenformen parallel ausgeben zu können. Die Frequenz beider Wellenformen ist gekoppelt und wird über einen einzelnen Potentiometer gesteuert, die Amplituden sind seperat anzusteuern.

\subsubsection{Benutzung}
\label{sec:orgbf6a7d2}

LFOs können für eine große Anzahl an Zwecken genutzt werden, der wohl simpelste davon ist wohl, das erzeugte Signal direkt als Audio auszugeben. Häufiger wird die Spannung eines LFOs als Kontrollspannung genutzt, beispielsweise als Trigger für einen Hüllkurvengenerator oder zum ansteuern eines VCAs um eine Funktion ähnlich eines Arpeggiators zu erfüllen.

\subsection{White Noise}
\label{sec:orgcadb3d7}
\url{https://www.youtube.com/watch?v=cyQMa4U0Wfs}
Noise, beziehungsweise Rauschen ist eine Art von Spannungssignal, welches auf eine nicht oder schwer vorherzusehende Art und Weise schwingt. Dabei entsteht ein Klang mit einer Vielzahl an Teilfrequenzen. White noise, beziehungsweise weißes Rauschen ist eine Art von Rauschen, bei welchem in einem kleinen Zeitraum alle Frequenzen in einem gegebenen Frequenzspektrum mit näherungsweise gleicher Amplitude vorhanden sind. Der Name entspringt einer Analogie zu sichtbarem Licht, als Beispiel deckt weißes Licht alle Frequenzen des Sichtbaren Lichtspektrum in gleicher Intensität ab. Eine weitere häufige Art von Rauschen ist pinkes Rauschen, bei welchem alle Frequenzen des hörbaren Spektrums abgedeckt werden, jedoch niedrigere Frequenzen in höherer Amplitude vorhanden sind.
\url{https://web.archive.org/web/20050720074308/https://mv.lycaeum.org/M2/noise\_ahf.html}

Um ein Gefühl für Rauschen zu bekommen, kann \url{https://mynoise.net/NoiseMachines/whiteNoiseGenerator.php} benutzt werden.
\subsubsection{Spezifikationen}
\label{sec:org555d54d}
\subsubsection{Elektronik}
\label{sec:org78c8434}
\subsubsection{Schematics}
\label{sec:org6c318b6}
\begin{circuitikz}[european]
\draw{
(0,0) node[ground, anchor=center, name=G]{} to[R, name=R1] ++(0,1)
};
\end{circuitikz}
\subsubsection{Benutzung}
\label{sec:orgae6fb92}
Weißes Rauschen kann für eine Vielzahl an Zwecken verwendet werden, beispielsweise als Kontrollspannung für einen VCA oder als Audiosignal. Weißes Rauschen kann dafür benutzt werden Rauschen anderer "Farben" zu erzeugen. Beispielsweise kann Pinkes Rauschen erzeugt werden, indem dem White Noise Modul ein Tiefpassfilter nachgeschalten wird.

\subsection{Mixer}
\label{sec:org49b0e26}
Die Aufgabe eines Mixers ist die Zusammenführug (das "mixen") von mehreren (in diesem Fall bis zu drei) Signalen. Unser Mixer Modul besitzt neben dem regulären auch einen invertierenden Output, welcher positive Spannungen als negative Spannungen mit dem selben Betrag ausgibt und umgekehrt.

\subsubsection{Spezifikationen}
\label{sec:org35da943}
Spannung: voller Spannungsbereich möglich (=> bis zu 24\href{file:///home/felixp/Documents/diplomarbeit/dokumentation/content/hauptteil.org}{Vpp})
\subsubsection{Elektronik}
\label{sec:org0387fe9}
\subsubsection{Schematics}
\label{sec:orgc16ef95}
\subsubsection{Benutzung}
\label{sec:orge46aeb7}
Die zu mixenden Signale werden durch die oberen drei Audiobuchsen angeschlossen. Die unterste Audiobuchse liefert das Ausgangssignal. Als einfachen Patch könnte man die beiden Signale des \href{modules/oscillator.org}{Oszillator x2} Moduls zusammenführen um beide Oszillatoren auf einmal zu hören.

\subsection{Attack-Release Hüllkurvengenerator}
\label{sec:orge837f8f}
\subsubsection{Spezifikationen}
\label{sec:org9ff1166}
\subsubsection{Elektronik}
\label{sec:org592b32a}
\subsubsection{Schematics}
\label{sec:org0c46003}
\subsubsection{Benutzung}
\label{sec:org8b785f5}

\subsection{Attack-Release Hüllkurvengenerator}
\label{sec:org2736dc9}
\subsubsection{Spezifikationen}
\label{sec:org3489da5}
\subsubsection{Elektronik}
\label{sec:org0022ca1}
\subsubsection{Schematics}
\label{sec:org312191d}
\subsubsection{Benutzung}
\label{sec:org6a42eb4}
