\chapter{Theoretische Grundlagen}

\section{Klangsynthese}
\label{sec:org29dddc4}
Das wort Synthese bedeutet in etwa zusammensetzen oder zusammenfügen \url{https://www.duden.de/rechtschreibung/Synthese}, beschreibt also das Erschaffen von etwas neuem durch die Vereinigung von kleineren Teilen, Klangsynthese bedeutet also, aus grundlegenden Klangwellen komplexere Klänge zu erzeugen.

\subsection{Klangwellen}
\label{sec:orgd71e4ed}
Klangwellen sind die Grundbausteine der Klangsynthese. Sie können auf verschiedene Arten und Weisen ausgedrückt werden und in verschiedenen Medien vorkommen, wichtig sind für uns vor allem als Spannung ausgedrückte Klangwellen, welche wir mit Elektronik manipulieren können und Schallwellen in der Luft als Endprodukt. Weitere Formen von Klangwellen sind zB die Schwingungen einer Gitarrensaite oder Lautsprechermembran oder Elektromagnetische Wellen, zB beim Funk.

Die Tonhöhe einer Klangwelle hängt von ihrer Frequenz, also von der Geschwindigkeit ab, in welcher sie schwingt. Dabei nimmt das menschliche Gehirn Töne auf eine logarithmische Art und Weise wahr, um eine große Bandbreite an Tonhöhen differenzieren zu können. Deshalb entspricht ein Ton mit der doppelten Frequenz dem selben Ton eine Oktave höher.

Die Lautstärke einer Klangwelle, also ihre Amplitude, wird ebenfalls auf eine logarithmische Art und Weise wahrgenommen. So bedeutet eine Steigerung der Lautstärke um 20 \si{\dB} eine verzehnfachung der Amplitude, also des Energieniveaus der Welle.

Klangwellen können verschiedene Formen besitzen, die grundlegende Form ist eine Sinuswelle. Das menschliche Ohr empfindet eine sinusförmige Schallwelle als "rein", da sie eine einzelne Frequenz repräsentiert. Weitere einfache Formen sind Rechteckswellen, Sägezahnwellen und Dreieckwellen. Diese Wellenformen besitzen neben ihrer Grundfrequenz weitere Obertöne, welche dem Klang Farbe verleihen.

Mithilfe einer Fourier-Transformation \url{https://mathworld.wolfram.com/FourierTransform.html} können alle Arten von Wellen durch eine Serie von Sinuswellen ausgedrückt werden. Durch einen Filter kann die Amplitude von Teilwellen unter oder über einer bestimmten Frequenz verringert werden, wodurch zB unangenehm hohe Obertöne gefiltert werden können. Filter sind ein einfaches beispiel für subtraktive Klangsynthese.

\subsection{Klangerzeugung}
\label{sec:orgdd49c80}
Die für die Klangsynthese benötigten Grundtöne können aus verschiedensten Quellen stammen. Die häufigste davon ist wohl ein Oszillator, welcher durchgehende Klangwellen mit einer einfachen Wellenform wie zb einem Sinus oder einer Rechteckswelle generiert. In einem Späteren Teil dieser Dokumentation wird Aufbau und Design eines solchen Oszillators beschrieben.

Abgesehen davon können eine breite Spanne von elektronischen Musikinstrumenten / Klangquellen wie E-Gitarren, Thereminen oder Radios und Kassettenspielern die zu modifizierenden Klangwellen für einen Synthesizer bereitstellen.

\section{Das Eurorack Format}
\label{sec:org44116e9}

1996 veröffentlichte Doepfer Musikelektronik GmbH eine Reihe an Synthesizermodulen. Schnell wurden kompatible Module von anderen Herstellern hergestellt, wodurch das Eurorack Format zum de-facto Standard für modulare Synthesizer wurde. Heute gibt es tausende von Eurorack Modulen, hergestellt von bekannten Firmen wie Moog, Roland, Behringer und auf Eurorack spezialisierten Herstellern wie Make Noise. Des weiteren gibt es eine lebendige DIY-szene mit vielen öffentlichen Designs, Anleitungen, Schematics, vorbereiteten Kits zum zusammenbauen und mehr.

Essentiell bei Eurorack Modulen ist, dass viele Funktionen nicht nur durch den Benutzer (durch Knöpfe, Potentiometer, etc) sondern auch durch andere Module mithilfe von sog. Kontrollspannung (CV) ansteuerbar sind. So kann z.B die Frequenz eines Oszillators, der Cutoff eines Filters, Attack und Releaselänge eines Envelopes usw. durch ein anderes Signal kontrolliert werden; Diese Kontrollspannung kann wiederum aus verschiedensten Modulen wie z.B. einem MIDI Interface, einem LFO, einem Envelope Generator wie zB ADSR oder sogar einem anderen Audiosignal kommen. Die Module sind nicht fest verkabelt, sondern werden vom Benutzer ``on the fly'' mit Patchkabeln (\SI{3.5}{\mm} mono) verbunden. Dadurch entsteht ein Netzwerk an elektronischen Schaltungen welche sich gegenseitig beeinflussen und hochschaukeln, was zu idealerweise wohlklingenden, jedoch in jedem Fall interessanten Effekten führt.

\subsection{Jargon}
\label{sec:org461286c}
\subsubsection{Frequenz}
\label{sec:org3fe41a6}
Die Frequenz einer Welle gibt an, wie schnell diese Welle schwingt, bzw wie oft in einem bestimmten Zeitraum sie ihren Kreislauf wiederholt. Die Einheit ist Hertz, \SI{1}{\hertz} entspricht 1/s, das bedeutet die Hertz Anzahl einer Welle gibt an, wie oft das Signal pro Sekunde schwingt. Die Frequenz einer Klangwelle entspricht ihrer Tonhöhe.

\subsubsection{Amplitude}
\label{sec:orgadd0ccc}
Die Amplitude einer Welle gibt an, wie hoch die Differenz zwischen den Höhepunkten und den Tiefpunkten dieser Welle ist. Die Einheit der Amplitude hängt vom Medium ab, in welchem die Welle schwingt, bei einer Spannungswelle die von einem Oszillator generiert wird, wird die Amplitude beispielsweise in Volt angegeben (bzw in \hyperref[sec:org549d6e7]{Vpp}). Die Amplitude einer Schallwelle entspricht ihrer Lautstärke, kann also in dB angegeben werden.

\subsubsection{Oberwellen}
\label{sec:org50de104}
Sind Klangwellen, welche einer gegebenen sinusförmigen Grundfrequenz Klangfarbe verleihen. Die Frequenzen dieser Oberwellen teilen die Grundfrequenz ganzzahlig, sind also Harmonisch mit dieser.

\subsubsection{Vpp}
\label{sec:org549d6e7}
Vpp steht für Voltage peak-to-peak, beschreibt also die Differenz zwischen Minimaler und Maximaler Spannung eines Signals. Wenn nicht anderst angegeben, sind Spannungen/Spannungsbereiche, welche in Vpp ausgedrückt sind, symmetrisch um 0V. Beispielsweise besitzt eine Spannungswelle von 10Vpp einen Spannungsbereich -5V bis +5V.

\subsubsection{Kontrollspannung, Control Voltage}
\label{sec:org5460bca}
Kontrollspannung (CV) ist die Quintessenz eines Modularen Synthesizers. Während normale Synthesizer wie der Minimoog intern mit Kontrollspannung arbeiten und oft auch Kontrollspannung ausgeben können (oder zumindest Audiospannung welche als Kontrollspannung misbraucht werden kann), sind die Leitungen für diese Kontrollspannungen fest verlötet. Das bedeutet dass der Benutzer nicht frei entscheiden kann, welcher Teil des Synthesizers welchen anderen Teil beeinflusst. Bei modularen Synthesizern liegen Audiobuchsen auf welchen Kontrollspannung anliegt / angelegt werden kann frei, diese Schnittstellen können vom Benutzer mehr oder weniger beliebig mit Patchkabeln zusammengeschlossen werden. Dadurch entsteht die Modularität des Eurorack Formats.

Kontrollspannungen sind im Regelfall entweder \SI{-2.5}{\volt} bis \SI{+2.5}{\volt} oder \SI{0}{\volt} bit \SI{8}{\volt}, können jedoch theoretisch den vollen möglichen Spannungsumfang von \SI{-12}{\volt} bis \SI{+12}{\volt} ausnutzen.

\subsubsection{Audiosignale}
\label{sec:org4652f65}
Audiosignale sind Spannungen zwischen -5V und +5V welche an einen Verstärker oder Lautsprecher angelegt werden können um Schall zu erzeugen. Sie können auch zur Weiterverarbeitung von einem Modul zum anderen geschickt werden und sogar als Kontrollspannung verwendet werden. Man kann Audiosignale als Kontrollspannungen, welche zum Ansteuern von Lautsprechern geeignet sind, sehen.

\subsubsection{Trigger}
\label{sec:org54091d8}
Auch bei einem analogen Synthesizer werden manchmal binäre Signale benötigt. Diese werden durch 5V (HIGH) bzw 0V (LOW) dargestellt. Aus diesem Grund (und für die Versorgung von zB Mikrocontrollern) wird eine eigene 5V Spannungsquelle vom Netzteil bereitgestellt.

\subsubsection{Patchkabel}
\label{sec:org74acb1e}
Patchkabel sind 3.5mm mono Klinkenstecker. Sie dienen dazu, Verbindungen zwischen verschiedenen Modulen herzustellen über welche Kontrollspannungen und Audiosignale übertragen werden können.

\subsubsection{Voltage Controlled Modules}
\label{sec:org74a028a}
Module, welche durch Kontrollspannung angesteuert werden, haben oft das präfix VC (Voltage Controlled) im Namen. Beispiele dafür sind VCOs (Voltage Controlled Oscillator) und VCAs (Voltage Controlled Amplifier).
