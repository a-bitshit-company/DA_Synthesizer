\chapter{Theoretische Grundlagen}

\section{Klangsynthese}
\label{sec:orgd028b37}
Das wort Synthese bedeutet in etwa zusammensetzen oder zusammenfügen \url{https://www.duden.de/rechtschreibung/Synthese}, beschreibt also das Erschaffen von etwas neuem durch die Vereinigung von kleineren Teilen, Klangsynthese bedeutet also, aus grundlegenden Klangwellen komplexere Klänge zu erzeugen.

Ein Synthesizer ist ein (meist elektronisches) Instrument, welches zur Klangsynthese fähig ist. Während die meisten herkömmlichen analogen Instrumente nur eine oder wenige unterschiedliche Klangfarben erzeugen können, ist eine der Kernaufgaben eines Synthesizers das Erzeugen von Klängen mit beliebig änderbaren Klangfarben. Zwar können Synthesizer auch herkömmliche Instrumente imitieren, vor allem sind sie jedoch das Mittel der Wahl, wenn unnatürliche und untypische Klangfarben gefragt sind, oder wenn sich die Klangfarbe eines Tones ändern soll, während er gespielt wird.

\subsection{Klangwellen}
\label{sec:org8745516}
Klangwellen sind die Grundbausteine der Klangsynthese. Sie können auf verschiedene Arten und Weisen ausgedrückt werden und in verschiedenen Medien vorkommen, wichtig sind für uns vor allem als Spannung ausgedrückte Klangwellen, welche wir mit Elektronik manipulieren können und Schallwellen in der Luft als Endprodukt. Weitere Formen von Klangwellen sind zum Beispiel die Schwingungen einer Gitarrensaite oder Lautsprechermembran oder Elektromagnetische Wellen, zum Beispiel beim Funk.

Die Tonhöhe einer Klangwelle hängt von ihrer Frequenz, also von der Geschwindigkeit ab, in welcher sie schwingt. Dabei nimmt das menschliche Gehirn Töne auf eine logarithmische Art und Weise wahr, um eine große Bandbreite an Tonhöhen differenzieren zu können. Deshalb entspricht ein Ton mit der doppelten Frequenz dem selben Ton eine Oktave höher.

Die Lautstärke einer Klangwelle, also ihre Amplitude, wird ebenfalls auf eine logarithmische Art und Weise wahrgenommen. So bedeutet eine Steigerung der Lautstärke um 20 \si{\dB} eine verzehnfachung der Amplitude, also des Energieniveaus der Welle.

Klangwellen können verschiedene Formen besitzen, die grundlegende Form ist eine Sinuswelle. Das menschliche Ohr empfindet eine sinusförmige Schallwelle als "rein", da sie eine einzelne Frequenz ohne Obertöne repräsentiert. Weitere einfache Formen sind Rechteckswellen, Sägezahnwellen und Dreieckwellen. Diese Wellenformen besitzen neben ihrer Grundfrequenz weitere Obertöne, welche dem Klang Farbe verleihen.

\subsection{Klangerzeugung}
\label{sec:org03d2400}
Die für die Klangsynthese benötigten grundlegenden Klangwellen können aus verschiedensten Quellen stammen. Die häufigste davon ist wohl ein Oszillator, welcher durchgehende Klangwellen mit einer einfachen Wellenform wie zum Beispiel einem Sinus oder einer Rechteckswelle generiert. In einem Späteren Teil dieser Dokumentation wird Aufbau und Design eines solchen Oszillators beschrieben.

Abgesehen davon können eine breite Spanne von elektronischen Musikinstrumenten / Klangquellen wie E-Gitarren, Thereminen oder Radios und Kassettenspielern die zu modifizierenden Klangwellen für einen Synthesizer bereitstellen.

\subsection{Subtraktive Klangsynthese}
\label{sec:org84d5f29}
Das Prinzip der Subtraktiven Klangsynthese besteht darin, Grundtöne mit vielen Obertönen zu filtern, um Töne mit einer gewünschten Klangfarbe zu erzeugen. Durch einen Filter wird die Amplitude von Teilwellen unter einer bestimmten Frequenz (=> high-pass Filter) oder über einer bestimmten Frequenz (=> low-pass Filter) verringert, wodurch zum Beispiel unangenehm hohe Obertöne gefiltert werden können.

Nach einem solchen Filter wird oft ein VCA geschalten, welcher die Amplitude des Eingangssignals proportional zur angelegten Kontrollspannung skaliert. Diese Kontrollspannung kann beispielsweise durch einen LFO oder Hüllkurvengenerator bereitgestellt werden. Dadurch kann einem durchgehenden Signal Rhythmus verliehen werden.

\subsection{Additive Klangsynthese}
\label{sec:orga01f8df}
Laut Fourier \url{https://mathworld.wolfram.com/FourierTransform.html} kann jegliche Art von Wellenform durch eine Serie von Sinuswellen ausgedrückt werden. Das Prinzip der additiven Klangsynthese besteht somit darin, eine Vielzahl von Sinuswellen mit unterschiedlichen Amplituden und Frequenzen zu Kombinieren, (beispielsweise durch einen Mixer) um Klänge mit jeglicher erdenklichen Klangfarbe zu erzeugen. Idealerweise wird jede grundlegende Sinuswelle durch eine seperate Hüllkurve moduliert um einen Klang mit laufend verändernder Klangfarbe zu erzeugen \url{http://www.raffaseder.com/sounddesign/klangsynthese/additiv.htm}. Da dies mit einer steigenden Anzahl an grundlegenden Sinuswellen eine technische Herausforderung darstellt, sind additive Synthesizer meist Digital ausgeführt, ein analoges Beispiel für einen additiven synthesizer wäre eine Orgel.

\subsection{Vocoder}
\label{sec:org086975e}
Ein vocoder basiert auf dem Prinzip, ein Signal (meist eine Stimme) mittels mehrerer Band-Pass Filter in seine Frequenzbestandteile aufzuteilen. Anschließend wird dieses Frequenzspektrum auf der Basis von weißem Rauschen wieder aufgebaut um einen als gesprochenes Wort zu erkennenden Klang zu erzeugen. Ein Vocoder arbeitet somit sowohl mit subtraktiver Soundsynthese bei der Analyse des Frequenzspektrums als auch mit additiver Soundsynthese beim wieder zusammensetzen des analysierten Klangs.

\section{Geschichte}
\label{sec:orgc2a9152}
Bereits im frühen 20. Jahrhundert wurden Elektronische Schaltkreise dazu benutzt, Klänge zu erzeugen. Damals noch mit Vakuumröhren statt Transistoren hergestellt, stellt das \textbf{Theremin} eines der ältesten heute noch verwendeten Elektronischen Musikinstrumente dar.

Der erste vollwertige elektronische Synthesizer, welcher auch als solcher bezeichnet wurde, war der RCA Music Synthesizer, eine raumhohe Maschine welche als Gemeinschaftsprojekt zwischen den amerikanischen Universitäten von Princeton und Columbia entstanden war. Statt mit einer Klaviertastatur spielte, beziehungsweise programmierte, man diesen Synthesizer erst mittels Lochkarten und konnte dann gewisse Aspekte des Klanges dynamisch während das Stück spielte ändern.

Das Konzept eines modularen Synthesizers und damit auch das Konzept der Kontrollspannung wurde erstmals von Robert Moog in seiner Arbeit mit dem Titel "VOLTAGE-CONTROLLED ELECTRONIC MUSIC MODULES" Dokumentiert. \url{https://moogfoundation.org/wp-content/uploads/AES-1964-No0320-Modules.pdf} Der Moog Modular Synthesizer, welcher auf diesen Prinzipien basiert, führte viele heute noch aktuelle Standards ein, wie zum Beispiel die Kontrollspannungsarten Trigger und \SI{1}{\volt} pro Oktave. Spätestens mit dem 1968 erschienenen Album "Switched-On Bach" wurde der Synthesizer als vollwertiges Instrument im Mainstream akzeptiert.

Während die Synthesizer von Moog mit dem Prinzip der Subtraktiven Klangsynthese arbeiteten, wurden zur gleichen Zeit, auf der anderen Seite Amerikas, erste Synthesizer mit additiver Klangsynthese hergestellt. Die von Donald Buchla hergestellten Synthesizer boten dem Benutzer beinahe grenzenlose Freiheit über die Farbe der erzeugten Klänge an. Dennoch blieb die Subtraktive Klangsynthese, wohl aufgrund größerer Intuitivität und besserer technischer Umsetzbarkeit das vorherrschende Prinzip.

Obwohl Moog als Vater der Modularen Klangsynthese gilt, ist eines der bekanntesten und beliebtesten Produkte der Firma Moog der fix verkabelte Minimoog. Dieser als live-Instrument gedachte Synthesizer führte ein Lautstärkerad und ein Tonhöhenveränderungsrad ein, mit welchem Töne ähnlich wie beim Saitenziehen bei einer Gitarre verändert werden können.

Die 70er und 80er Jahre waren vor allem von digitalen Synthesizern geprägt. Das von der Firma "New England Digital" hergestellte Synclavier I war der erste Synthesizer welcher Frequenzmodulation, ein Beispiel für additive Klangsynthese, anbot, der von Yamaha hergestellte DX7, brachte dieses Konzept in den Mainstream. Die Glockenartigen Klänge welche charakteristisch für diese Art der Klangsynthese sind, prägten den Großteil der 80er Jahre und sind auch heute noch häufig im Pop und im Schlager zu finden.

Das Konzept der Modularen Synthesizer schien beinahe vergessen, bis im Jahre 1992 Dieter Döpfer, gemeinsam mit der Band Kraftwerk das modulare Synthesizersystem A-100 entwarf. Die quelloffene Natur dieses Systems ermöglichte es anderen Herstellern wie auch der Firma Moog kompatible Module herzustellen, wodurch ein de-facto Standard entstand, heute bekannt als Eurorack, was zu einer renaissance der modularen Synthesizer führte.

Die Dokumentation für diesen Synthesizer, den A-100, stellt auf direkte oder indirekte Weise die Grundlage für die meisten Aspekte des in dieser Dokumentation beschriebenen Systems dar.

\section{Das Eurorack Format}
\label{sec:org90ff5b7}

1996 veröffentlichte Doepfer Musikelektronik GmbH eine Reihe an Synthesizermodulen. Schnell wurden kompatible Module von anderen Herstellern hergestellt, wodurch das Eurorack Format zum de-facto Standard für modulare Synthesizer wurde. Heute gibt es tausende von Eurorack Modulen, hergestellt von bekannten Firmen wie Moog, Roland, Behringer und auf Eurorack spezialisierten Herstellern wie Make Noise. Des weiteren gibt es eine lebendige DIY-szene mit vielen öffentlichen Designs, Anleitungen, Schematics, vorbereiteten Kits zum zusammenbauen und mehr.

Essentiell bei Eurorack Modulen ist, dass viele Funktionen nicht nur durch den Benutzer (durch Knöpfe, Potentiometer, etc) sondern auch durch andere Module mithilfe von sog. Kontrollspannung (CV) ansteuerbar sind. So kann z.B die Frequenz eines Oszillators, der Cutoff eines Filters, Attack und Releaselänge eines Envelopes usw. durch ein anderes Signal kontrolliert werden; Diese Kontrollspannung kann wiederum aus verschiedensten Modulen wie z.B. einem MIDI Interface, einem LFO, einem Envelope Generator wie zum Beispiel ADSR oder sogar einem anderen Audiosignal kommen. Die Module sind nicht fest verkabelt, sondern werden vom Benutzer ``on the fly'' mit Patchkabeln (\SI{3.5}{\mm} mono) verbunden. Dadurch entsteht ein Netzwerk an elektronischen Schaltungen welche sich gegenseitig beeinflussen und hochschaukeln, was zu idealerweise wohlklingenden, jedoch in jedem Fall interessanten Effekten führt.


\subsection{Kontrollspannung}
\label{sec:org2965e4e}
Das Konzept der Kontrollspannung ist grundlegend für jegliche Art von analoger Klangsynthese. Der Zweck von Kontrollspannung ist, bestimmte Parameter von Modulen, welche Kontrollspannung akzeptieren, anzusteuern. Beispielsweise akzeptieren manche Oszillatoren Kontrollspannung zum einstellen der Frequenz. Solche Oszillatoren können von einem weiteren Oszillator mit niedriger Frequenz (LFO) angesteuert werden um einen Vibrato-Effekt zu erzeugen.

Besonders für Eurorack und für modulare Synthesizer im Generellen hat dieses Konzept einen hohen Stellenwert, da bei solchen Systemen Audiosignale und Kontrollspannungen nicht fix verkabelt sind, sondern vom Benutzer je nach Bedarf geschalten werden können. Es können sogar Audiosignale als Kontrollspannung benutzt werden, wodurch die Unterscheidung dieser beiden Arten von Spannung etwas an Bedeutung verliert. Es gibt verschiedene Arten von Kontrollspannung, welche sich je nach Verwendungszweck unterscheiden:

\subsubsection{Hüllkurve}
\label{sec:orgd5692b8}
\url{https://making-music.com/quick-guides/envelopes/}
Hüllkurven besitzen meist einen positiven Spannungsbereich. Sie werden oft zum Ansteuern von Spannungskontrollierbaren Verstärkern (VCA) benutzt. Hüllkurven treten oft in Form von ADSR (Attack, Decay, Sustain, Release) auf, diese Art von Hüllkurve beschreibt den Verlauf der Lautstärke eines Tones beim Drücken einer Taste.

\begin{enumerate}
\item Attack
\label{sec:orgac1bf99}
Der "Attack" Wert gibt an, wie lange der Ton nach dem Drücken der Taste braucht, um auf seine maximale Lautstärke anzuschwellen.

\item Decay
\label{sec:orgefac98d}
Nachdem der Ton seine maximale Lautstärke erreicht hat, schwillt er auf eine niedrigere Lautstärke ab. Der Decay-Wert, gibt an, wie lange der Ton benötigt, um diese niedrigere Lautstärke zu erreichen.

\item Sustain
\label{sec:org706f59c}
Im Unterschied zu den Anderen Parametern ist der Sustain-Wert eine Amplitude anstatt einer Zeit. Der eingestellte Wert gibt an, auf welche Lautstärke das Signal nach dem Ablaufen der Decay-Zeit abschwillt. Die eingestellte Lautstärke ist konstant, solange die Taste gedrückt bleibt.

\item Release
\label{sec:org0a77f40}
Nach dem Loslassen der Taste benötigt der Ton eine gewisse Zeit, um eine Lautstärke von 0 zu erreichen. Diese Zeit wird über den Release-Parameter eingestellt.
\end{enumerate}

\subsubsection{Trigger}
\label{sec:org9d34642}
\subsubsection{Audio}
\label{sec:org85de0d2}

\subsection{Jargon}
\label{sec:orge6a0856}
\subsubsection{Frequenz}
\label{sec:orgc5d5d54}
Die Frequenz einer Welle gibt an, wie schnell diese Welle schwingt, bzw wie oft in einem bestimmten Zeitraum sie ihren Kreislauf wiederholt. Die Einheit ist Hertz, \SI{1}{\hertz} entspricht 1/s, das bedeutet die Hertz Anzahl einer Welle gibt an, wie oft das Signal pro Sekunde schwingt. Die Frequenz einer Klangwelle entspricht ihrer Tonhöhe.

\subsubsection{Amplitude}
\label{sec:org29b58e7}
Die Amplitude einer Welle gibt an, wie hoch die Differenz zwischen den Höhepunkten und den Tiefpunkten dieser Welle ist. Die Einheit der Amplitude hängt vom Medium ab, in welchem die Welle schwingt, bei einer Spannungswelle die von einem Oszillator generiert wird, wird die Amplitude beispielsweise in Volt angegeben (bzw in \hyperref[sec:orgf9e62e2]{Vpp}). Die Amplitude einer Schallwelle entspricht ihrer Lautstärke, kann also in dB angegeben werden.

\subsubsection{Oberwellen}
\label{sec:org91b64a4}
Sind Klangwellen, welche einer gegebenen sinusförmigen Grundfrequenz Klangfarbe verleihen. Die Frequenzen dieser Oberwellen teilen die Grundfrequenz ganzzahlig, sind also Harmonisch mit dieser.

\subsubsection{Vpp}
\label{sec:orgf9e62e2}
Vpp steht für Voltage peak-to-peak, beschreibt also die Differenz zwischen Minimaler und Maximaler Spannung eines Signals. Wenn nicht anderst angegeben, sind Spannungen/Spannungsbereiche, welche in Vpp ausgedrückt sind, symmetrisch um 0V. Beispielsweise besitzt eine Spannungswelle von 10Vpp einen Spannungsbereich -5V bis +5V.

\subsubsection{Kontrollspannung, Control Voltage}
\label{sec:orge7569ea}
Kontrollspannung (CV) ist die Quintessenz eines Modularen Synthesizers. Während normale Synthesizer wie der Minimoog intern mit Kontrollspannung arbeiten und oft auch Kontrollspannung ausgeben können (oder zumindest Audiospannung welche als Kontrollspannung misbraucht werden kann), sind die Leitungen für diese Kontrollspannungen fest verlötet. Das bedeutet dass der Benutzer nicht frei entscheiden kann, welcher Teil des Synthesizers welchen anderen Teil beeinflusst. Bei modularen Synthesizern liegen Audiobuchsen auf welchen Kontrollspannung anliegt / angelegt werden kann frei, diese Schnittstellen können vom Benutzer mehr oder weniger beliebig mit Patchkabeln zusammengeschlossen werden. Dadurch entsteht die Modularität des Eurorack Formats.

Kontrollspannungen sind im Regelfall entweder \SI{-2.5}{\volt} bis \SI{+2.5}{\volt} oder \SI{0}{\volt} bit \SI{8}{\volt}, können jedoch theoretisch den vollen möglichen Spannungsumfang von \SI{-12}{\volt} bis \SI{+12}{\volt} ausnutzen.

\subsubsection{Audiosignale}
\label{sec:org600b5cb}
Audiosignale sind Spannungen zwischen -5V und +5V welche an einen Verstärker oder Lautsprecher angelegt werden können um Schall zu erzeugen. Sie können auch zur Weiterverarbeitung von einem Modul zum anderen geschickt werden und sogar als Kontrollspannung verwendet werden. Man kann Audiosignale als Kontrollspannungen, welche zum Ansteuern von Lautsprechern geeignet sind, sehen.

\subsubsection{Trigger}
\label{sec:org56bae3a}
Auch bei einem analogen Synthesizer werden manchmal binäre Signale benötigt. Diese werden durch 5V (HIGH) bzw 0V (LOW) dargestellt. Aus diesem Grund (und für die Versorgung von Mikrocontrollern) wird eine eigene 5V Spannungsquelle vom Netzteil bereitgestellt.

\subsubsection{Patchkabel}
\label{sec:org2ca656a}
Patchkabel sind 3.5mm mono Klinkenstecker. Sie dienen dazu, Verbindungen zwischen verschiedenen Modulen herzustellen über welche Kontrollspannungen und Audiosignale übertragen werden können.

\subsubsection{Voltage Controlled Modules}
\label{sec:orgd6b6039}
Module, welche durch Kontrollspannung angesteuert werden, haben oft das präfix VC (Voltage Controlled) im Namen. Beispiele dafür sind VCOs (Voltage Controlled Oscillator) und VCAs (Voltage Controlled Amplifier).
